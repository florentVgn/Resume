%%%%%%%%%%%%%%%%%
% This is an sample CV template created using altacv.cls
% (v1.1.5, 1 December 2018) written by LianTze Lim (liantze@gmail.com). Now compiles with pdfLaTeX, XeLaTeX and LuaLaTeX.
%
%% It may be distributed and/or modified under the
%% conditions of the LaTeX Project Public License, either version 1.3
%% of this license or (at your option) any later version.
%% The latest version of this license is in
%%    http://www.latex-project.org/lppl.txt
%% and version 1.3 or later is part of all distributions of LaTeX
%% version 2003/12/01 or later.
%%%%%%%%%%%%%%%%

%% If you need to pass whatever options to xcolor
\PassOptionsToPackage{dvipsnames}{xcolor}
\PassOptionsToPackage{lowtilde}{url}

%% If you are using \orcid or academicons
%% icons, make sure you have the academicons
%% option here, and compile with XeLaTeX
%% or LuaLaTeX.
% \documentclass[10pt,a4paper,academicons]{altacv}

%% Use the "normalphoto" option if you want a normal photo instead of cropped to a circle
% \documentclass[10pt,a4paper,normalphoto]{altacv}

\documentclass[11pt, a4paper, ragged2e]{altacv}

\usepackage{hyperref}

%% AltaCV uses the fontawesome and academicon fonts
%% and packages.
%% See texdoc.net/pkg/fontawecome and http://texdoc.net/pkg/academicons for full list of symbols. You MUST compile with XeLaTeX or LuaLaTeX if you want to use academicons.

% Change the page layout if you need to
\geometry{left=1cm,right=9cm,marginparwidth=7cm,marginparsep=1cm,top=0.5cm,bottom=0.5cm}

% Change the font if you want to, depending on whether
% you're using pdflatex or xelatex/lualatex
\ifxetexorluatex
  % If using xelatex or lualatex:
  \setmainfont{Carlito}
\else
  % If using pdflatex:
  \usepackage[utf8]{inputenc}
  \usepackage[T1]{fontenc}
  \usepackage[default]{lato}
\fi

% Change the colours if you want to
\definecolor{IFBlue}{HTML}{020a40}
\definecolor{IFDarkBlue}{HTML}{00425C}
\definecolor{IFMediumBlue}{HTML}{17739A}
\definecolor{IFLightBlue}{HTML}{0d54b1}

\definecolor{IFGrey}{HTML}{30103a}
\definecolor{IFDarkGrey}{HTML}{231F20}
\definecolor{IFMediumGrey}{HTML}{4D4D4F}
\definecolor{IFLightGrey}{HTML}{808285}

\colorlet{heading}{IFBlue}
\colorlet{accent}{IFLightBlue}
\colorlet{emphasis}{IFMediumGrey}
\colorlet{body}{IFDarkGrey}

% Change the bullets for itemize and rating marker
% for \cvskill if you want to
\renewcommand{\itemmarker}{{\small\textbullet}}
\renewcommand{\ratingmarker}{\faCircle}

%% sample.bib contains your publications
\addbibresource{sample.bib}

\begin{document}
\name{Florent Viogne}
\tagline{Futur ingénieur diplômé à la recherche de son premier CDI en ingénierie logicielle}
\photo{3cm}{photo2}
\personalinfo{%
  % Not all of these are required!
  % You can add your own with \printinfo{symbol}{detail}
  \email{\href{mailto:viognef@gmail.com}{viognef@gmail.com}}
  \phone{\href{tel:0616845475}{06 16 84 54 75}}
  \location{106 cours Berriat, 38000 Grenoble}
  \linkedin{\href{https://linkedin.com/in/florent-viogne}{florent-viogne}}
  \github{\href{https://github.com/florentVgn}{florentVgn}}
  \
  %\homepage{\url{http://fma.if.usp.br/~nickolas}}
  %% You MUST add the academicons option to \documentclass, then compile with LuaLaTeX or XeLaTeX, if you want to use \orcid or other academicons commands.
  % \orcid{orcid.org/0000-0000-0000-0000}
}

%% Make the header extend all the way to the right, if you want.
\begin{fullwidth}
\makecvheader
\end{fullwidth}

%% Depending on your tastes, you may want to make fonts of itemize environments slightly smaller
% \AtBeginEnvironment{itemize}{\small}

%% Provide the file name containing the sidebar contents as an optional parameter to \cvsection.
%% You can always just use \marginpar{...} if you do
%% not need to align the top of the contents to any
%% \cvsection title in the "main" bar.
\cvsection[page1sidebar]{Expériences}

\cvevent{Apprenti ingénieur}{Orange Labs}{Sept. 2018 -- août 2021}{Meylan, Isère}{}{}

\begin{itemize}
    \item Conception et mise en \oe{}uvre de solution d’intégration continue et de collecte de métriques d’infrastructures cloud.
    \item Projet de fin d'étude : conception et mise en \oe{}uvre d'un détecteur de pannes, à partir des métriques du système et des applications, en utilisant l'apprentissage automatique. 
\end{itemize}
\divider

\cvevent{Développeur fullstack Java/Javascript}{Decalog}{Été 2018}{Guilherand-Granges, Ardèche}{}{}
\begin{itemize}
    \item Participation à la conception et au développement d'un système de gestion de bibliothèques en environnement agile.
\end{itemize}
\divider

\cvevent{Stagiaire développeur fullstack Java/Javascript}{Decalog}{Avril. 2018 -- juin 2018}{Guilherand-Granges, Ardèche}{}{}
\begin{itemize}
    \item Développement d’un système d’import d’abonnés dans un système de gestion de bibliothèques en environnement agile.
\end{itemize}
\divider

\cvevent{Employé saisonnier}{}{Été 2016 et 2017}{Ardèche}{}{}
\begin{itemize}
    \item 2017 : Mise en rayon libre-service à Hyper U (Alissas, Ardèche).
    \item 2016 : Cariste à Rhoda Coop (Tournon-sur-Rhône, \newline Ardèche).
\end{itemize}

\cvsection{Formation}

\cvevent{Diplôme d'ingénieur de l'ENSIMAG, filière ingénierie des systèmes d'informations}{Grenoble INP - ENSIMAG}{2018 -- 2021}{Grenoble, Isère}{}{}
\divider

\cvevent{DUT informatique}{IUT de Valence}{2016 -- 2018}{Valence, Drôme}{}{}
\divider

\cvevent{Classe préparatoire MPSI}{Lycée Champollion}{2015 -- 2016}{Grenoble, Isère}{}{}


\end{document}
